\chapter{Segal spaces and Segal categories}

\chapterprecistoc{\textup{by} Yan Zhao}

\begin{refsection}


In this talk, we discuss two models for $(\infty,1)$-categories in detail, namely complete Segal spaces and Segal categories. These two models are both based on a construction on simplicial spaces satisfying the Segal condition (\ref{eq seg con}), but with different technical conditions. A key advantage of these two models over other models such as quasi-categories is the existence of an explicit homotopy theory. Furthermore, the construction is intrinsic, allowing for iterated definitions of the homotopy theory of homotopy theories. As such, they are widely used to define the $(\infty,1)$-category of topological spaces or $\infty$-groupoids. We will demonstrate the construction of an $(\infty,1)$-enhancement of simplicial model categories in Sec.~\ref{egss}.

These two models, like quasi-categories, are examples of ``weak" $(\infty,1)$-categories, in the sense that composition (and thus associativity and identity) is only defined up to homotopy. This is in contrast with simplicial categories, which are ``strict" $(\infty,1)$-categories. While the notions of weak and strict are equivalent for $(\infty,1)$-categories, they are not for $(\infty,n)$-categories where $n\ge2$. Indeed, the definition of complete Segal spaces and Segal categories can easily be extended to $(\infty,n)$ structures (see, for example, \cite{bergner-rezk-comparison-models-oo-n-categories-I,hirschowitz-simpson-descente-pour-les-n-champs,lurie-classification-topological-field-theories}).

Complete Segal spaces come with a natural topology, and are hence often used in topological constructions such as bordism categories \cite{lurie-classification-topological-field-theories}. Both complete Segal spaces and Segal categories are widely used in derived algebraic geometry, for example in defining higher stacks \cite{hirschowitz-simpson-descente-pour-les-n-champs}.

\begin{flushright}
Yan Zhao
\end{flushright}

\section{Preliminaries}
\subsection{Simplicial space}
In this text, we shall take a \textbf{space} to be a simplicial set\footnote{It is also possible to work with topological spaces (or specifically, compactly generated Hausdorff spaces), eg. see \cite{lurie-classification-topological-field-theories}. However, some technical assumptions are necessary and many arguments are messier, for example, we need to assume that the simplicial space is cofibrant (which is automatic for spaces being simplicial sets). In line with the combinatorial flavour of these talks, we will stick to simplicial sets.}. Let $\sset$ be the category of spaces, which we endow with the Quillen model category structure (Thm.~\ref{thm model structure on sset}). 

A \textbf{simplicial space} is a functor $X:\SDelta^{op}\to\sset$, which sends $[n]\mapsto X_n$, with face and degeneracy maps $d_i:X_n\to X_{n-1}$ and $s_i:X_n\to X_{n+1}$. It can also be seen as a bisimplicial set $X:\SDelta^{op}\times\SDelta^{op}\to \Set$ with two sets of arrows $d_i$ and $s_i$. There is an obvious equivalence between the category of simplicial spaces $\ssp$ and that of bisimplicial sets $\sset^{(2)}$. The former characterization is more convenient in the discussion of homotopy theory while the latter in the comparison theorems between different models of $\infty$-categories. We will freely interchange between the two characterizations.

We identify $\sset$ as a full subcategory of $\ssp$ by sending each simplicial set $K$ to the constant simplicial space $[n]\mapsto K$ with $d_i$ and $s_i$ being the identity maps. The category $\ssp$ can be enriched over simplicial sets in a compatible way with the enrichment of $\sset$. For any $X,Y\in \ssp$, we have the function complex $\Map_{\ssp}(X,Y)$, where each $n$-simplex is the set of simplicial maps $X\times\Delta^n\to Y$.

Let $F(k)$ be the simplicial space defined by $[n]\mapsto \SDelta([n],[k])$. where $\SDelta([n],[k])$ is taken as a discrete simplicial set. $F(k)$ is a $k$-th space functor, in the sense that there is a natural isomorphism $\Map_{\ssp}(F(k),X)\cong X_k$. With this notation, we can define an internal hom-object in $\ssp$ by
$$(Y^X)_k=\Map_{\ssp}(X\times F(k),Y).$$
It satisfies the natural isomorphism $\ssp(X\times Y,Z)\cong\ssp(X,Z^Y)$.

Let $\partial F(k)$ be the largest subobject of $F(k)$ not containing the identity map $\iota:[k]\to[k]$. It is generated by the faces $d_i\iota\in\SDelta([k-1],[k])$. We denote by $\partial X_k$ the mapping space $\Map_{\ssp}(\partial F(k),X)$. The inclusion $\partial F(k)\hookrightarrow F(k)$ induces a map $X_k\to\partial X_k$.


\subsection{Reedy model category structure on $\ssp$}
We can endow $\ssp$ with a model structure derived from the standard model structure on $\sset$. It is known as the Reedy model structure (see \cite[IV.3.2]{goerss-jardine-simplicial-homotopy-theory} for more details), and satisfies:
\begin{enumerate}
\item weak equivalences if $f_k:X_k\to Y_k$ are degree-wise weak equivalences;
\item cofibrations if $f_k$ are degree-wise cofibrations;
\item fibrations if the induced maps $X_k\to Y_k\times_{\partial Y_k}\partial X_k$ are fibrations.
\end{enumerate}
Note that all simplicial spaces are cofibrant and a simplicial space $X$ is Reedy fibrant iff $X_k\to\partial X_k$ is a fibration for each $k\ge 0$. Since Reedy weak equivalences are degreewise, a Reedy fibrant replacement functor is simply the fibrant replacement functor degreewise on each space $X_k$.

The Reedy model category structure on $\sset$ is well-behaved in the following senses: it is cofibrantly generated (see \cite{hirschhorn-model-categories} for definition), compatible with cartesian closure and proper. We omit the definitions, but instead state some useful properties, see, e.g., \cite{hirschhorn-model-categories}, for details.  

Cofibrant generation allows us to use the small object argument. Compatibility with cartesian closure means that given any cofibration (inclusion) $A\hookrightarrow B$ and $X$ fibrant, we have a fibration $\Map_{\ssp}(B,X)\to\Map_{\ssp}(A,X)$, which is trivial if $A\hookrightarrow B$ is. Properness implies that the pushout along a cofibration is a homotopy pushout and the pullback along a fibration is a homotopy pullback \cite[Prop A.2.2.4]{lurie-higher-topos-theory}.

\section{Segal spaces}
In this section, we will construct our first model of $(\infty,1)$-categories, the complete Segal spaces. A complete Segal space $W$ is a simplicial space, with some additional conditions imposed. Intuitively, as a model for $(\infty,1)$-categories, the objects are given by the 0-space $W_0$, while 1-morphisms are given by points in $W_1$. Homotopies between morphisms are described explicitly by the higher spaces $W_i$ for $i>1$. As such, complete Segal spaces have an explicit homotopy structure, which Rezk described as the study of homotopy theory of homotopy theories. For most of this section, we follow the explicit constructions given by Rezk \cite{rezk-a-model-for-the-homotopy-theory-of-homotopy-theories}.

\subsection{The Segal conditon}
The main condition in defining a complete Segal space from a simplicial space is the Segal condition. As all higher morphisms in an $(\infty,1)$-category are invertible (up to homotopy), we need to impose a restriction on the higher simplices. The condition is that $W_n$ has to be naturally weakly equivalent to $(W_1)^n$.

Formally, let $G(k)\subset F(k)$ be the simplicial subspace generated by the segments $[i,i+1]\in F(k)_1$, then
\begin{equation}\label{eq seg con}
\Map_{\ssp}(G(k),W)\cong\holim(W_1\xrightarrow{d_0}W_0\xleftarrow{d_1}W_1\xrightarrow{d_0}\cdots\xrightarrow{d_0}W_0\xleftarrow{d_1}W_1)
\end{equation}
with $k$ copies of $W_1$. The inclusion $\phi^k:G(k)\hookrightarrow F(k)$ induces a map $\phi_k=\Map_{\ssp}(\phi^k,W):W_k\to\Map_{\ssp}(G(k),W)$. 

\begin{defin}
A simplicial space $W$ satisfies the Segal condition if $\phi_k:W_k\to\Map_{\ssp}(G(k),W)$ is a weak equivalence for each $k\ge 2$.
\end{defin}

Therefore, simplicial spaces satisfying the Segal condition are defined up to weak equivalence by their 0-th and 1-st spaces.

We can define a simplicial enriched structure associated to a simplicial space satisfying the Segal condition:
\begin{defin}\label{simpcat}
Let $W$ be a simplicial space. Define $\Ob W=(W_0)_0$ to be the vertices of $W_0$ and for any $x,y\in\Ob W$, we define $\map_W(x,y)$ to be the fiber of the map $(d_1,d_0):W_1\to W_0\times W_0$ at the point $(x,y)\in W_0\times W_0$. The identity map is defined to be $\id_x=s_0x\in\Map_W(x,x)$.
\end{defin}
However, for a general simplicial space $W$, it is difficult to verify the Segal condition since (\ref{eq seg con}) is given by a homotopy limit. We want to use properness of the category of simplicial sets to simplify computations of homotopy limits. To do so we require some fibrancy conditions. $\map_W(x,y)$ is in general not fibrant, but simply imposing fibrancy on it (eg. by requiring $(d_1,d_0):W_1\to W_0\times W_0$ to be a fibration) is still insufficient. We also want homotopy relations to be well-defined up to changing of the end points by a path in $(W_0)_1$, that is, if $[x],[y]\in\pi_0(W_0)$ are the path-components of $x$ and $y$, then $\map_W([x],[y])$ to well-defined. It turns out that imposing Reedy fibrancy is sufficient\footnote{It is also possible to define a Segal space without the technical condition of Reedy fibrancy. See, for example, \cite{zhao-memoire} for one such discussion.}.

\begin{defin}
A Segal space is a Reedy fibrant simplicial space satisfying the Segal condition.
\end{defin}
If $W$ is Reedy fibrant, then (\ref{eq seg con}) can be computed as a limit (not just a homotopy limit) and the map $\phi_k$ is a fibration for all $k$. Similarly, $d_0,d_1:W_1\to W_0$ are fibrations, hence $(d_1,d_0):W_1\to W_0\times W_0$ is a fibration and $\map_W(x,y)$ is fibrant. Indeed, since $W_0$ is fibrant, the Segal condition ensures that $W_k$ is fibrant for all $k$. 

\subsection{Homotopy theory in Segal spaces}
In an $(\infty,1)$-category, we are only interested in maps up to homotopy and want to be able to compose maps up to homotopy. We shall now see how this can be done. Let $W$ be a Segal space and $x,y\in\Ob W$. We want to say that two maps $f,g\in\map(x,y)$ are \textbf{homotopic} (denoted $f\sim g$) if they lie in the same path component of $\map(x,y)$, i.e., $[f]=[g]\in\pi_0\map(x,y)$.

Taking the fibre of the Segal map \ref{eq seg con} at a point $(x_0,\ldots,x_k)\in W_0^{k+1}$ gives a fibration
$$\phi_k:\map(x_0,\ldots,x_k)\to\map(x_0,x_1)\times\cdots\times\map(x_{n-1},x_n).$$
The Segal condition ensures that this is a trivial fibration. Thus given any set of maps $f_i\in\map(x_{i-1},x_i)$, there exists $h\in\map(x_0,\ldots,x_k)$ such that $\phi_k(h)\sim (f_1,\ldots,f_k)$. Taking the projection (via face maps) of $h$ to $\map(x_0,x_k)$ gives the composition $f_k\circ\cdots\circ f_1$ as shown in the following diagram
\centerline{
\xymatrix{
\map(x_0,\ldots,x_k)\ar[r]^-{\phi_k}_{\sim}\ar[d]_{d_{1,\ldots,k-1}}&\map(x_0,x_1)\times\cdots\times\map(x_{n-1},x_n)\ar@{-->}[dl]^{\text{composition}}\\
\map(x_0,x_k)
}}

We can thus associate an ordinary category to each Segal space:
\begin{prop}[{\cite[Prop 5.4]{rezk-a-model-for-the-homotopy-theory-of-homotopy-theories}}]
For each Segal space $W$, we have an associated homotopy category $\Ho W$ where $\Ob\Ho W=\Ob W$ and $\map_{\Ho W}(x,y)=\pi_0\map_W(x,y)$.
\end{prop}

With these notions of homotopy and composition, we can define \textbf{homotopy equivalence} as usual: a map $g\in\map_W(x,y)$ is a homotopy equivalence if there exist $f,h\in\map_W(y,x)$ such that $g\circ f\sim\id_y$ and $h\circ g\sim\id_x$. In a more intrinsic language, let $Z(3)=(0\to2\leftarrow 1\to 3)\subset F(3)$ be a subsimplicial space, then $g$ is a homotopy equivalence if $(\id_x,g,\id_y)\in\Map_{\ssp}(Z(3),W)$ admits a lift to some $H\in W_3$.

Rezk \cite[Lemma 5.8]{rezk-a-model-for-the-homotopy-theory-of-homotopy-theories} showed that homotopy equivalence is a equivalence relation, so we can define the \textbf{space of homotopy equivalence} to be the union of the components $W_{\mathrm{hoequiv}}\subset W_1$ of homotopy equivalences. Note that $s_0:W_0\to W_1$ factors through $W_{\mathrm{hoequiv}}$ since $s_0x=\id_x\in W_{\mathrm{hoequiv}}$ for all $x\in W_0$.

Note that the functor $W\mapsto W_{\mathrm{hoequiv}}$ is representable, which allows a more intrinsic treatment of the space of homotopy equivalence. Let $E=(\bullet\rightleftarrows\bullet)=\mathrm{discnerve}(I[1])$ (see definition in examples). Rezk proved the following technical theorem:
\begin{thm}[{\cite[Thm 6.2]{rezk-a-model-for-the-homotopy-theory-of-homotopy-theories}}]\label{rephoequiv}
The map $\Map_{\ssp}(E,W)\to W_1$ induced by the inclusion $F(1)\hookrightarrow E$ factors through $W_{\mathrm{hoequiv}}\subset W_1$, and induces a weak equivalence $\Map_{\ssp}(E,W)\to W_{\mathrm{hoequiv}}$.
\end{thm}

\subsection{Complete Segal spaces}
There is a natural topology on a Segal space given by the space $W_0$. On an $(\infty,1)$-category, there is another natural topology given by taking the maximal subgroupoid of the category (i.e. the maximal sub-$(\infty,0)$-category. On a Segal space, this is clearly generated by $W_{\mathrm{hoequiv}}$. It is natural that we want these two topology to be the same. Indeed, imposing this additional structure is necessary to obtain a good category of $(\infty,1)$-categories from Segal spaces (see the following section on model structures). We thus define
\begin{defin}
A complete Segal space is a Segal space such that the map $s_0:W_0\to W_{\mathrm{hoequiv}}$ is a weak equivalence.
\end{defin}

Using Thm.~\ref{rephoequiv}, we have the following equivalent definitions:
\begin{cor}\label{rephoequivcor}
Let $W$ be a Segal space. The following are equivalent:
\begin{enumerate}
\item $W$ is a complete Segal space.
\item The map $W_0\to\Map_{\ssp}(E,W)$ induced by $E\to F(0)$ is a weak equivalence.
\item For each pair $x,y\in\Ob W$, the fibre $\mathrm{hoequiv}(x,y)$ of the fibration $W_{\mathrm{hoequiv}}\xrightarrow{(d_1,d_0)} W_0\times W_0$ is naturally weak equivalent to the space of paths in $W_0$ from $x$ to $y$.
\end{enumerate}
\end{cor}

\begin{cor}
Let $\Ob W/\sim$ denote the set of homotopy equivalence classes of objects in $\Ho W$. If $W$ is a complete Segal space, then $\pi_0W_0\cong\Ob W/\sim$.
\end{cor}
\begin{proof}
Follows immediately from (iii) of the previous corollary.
\end{proof}

We want to obtain a complete Segal space from every Segal space. Define the completion of $W$ to be a complete Segal space $\hat W$ with a map $i_W:W\to\hat W$ which is universal among all maps from $W$ to a complete Segal space. The following proposition shows that they exist and the construction is functorial.

\begin{prop}
For any Segal space $W$, there exists a functorial completion map given by $i_W:W\to\hat W$.
\end{prop}
\begin{proof}
We shall only give the construction of $i_W$, the details of the proof can be found in \cite[Sec.~14]{rezk-a-model-for-the-homotopy-theory-of-homotopy-theories}.Let $E(m)=\mathrm{discnerve}(I[m])$. For each $n \ge 0$, we can define a simplicial set $\tilde W_n=\diag([m]\mapsto (W^{E(m)})_n \cong \Map_{\ssp}(E(m)\times F(n),W))$ where the diagonal map $\diag \colon \ssp\cong\sset^{(2)}\to\sset$ is that induced by $[n]\mapsto[n]\times[n]$. The face and degeneracy maps induced from $d^i \colon F(n)\to F(n+1)$ and $s^i \colon F(n)\to F(n-1)$ gives us a simplicial space $\tilde W$ with a natural map $W\to \tilde W$. $\hat W$ is defined to be a functorial Reedy fibrant replacement of $\tilde W$, thus inducing a map $i_W:W\to\tilde W\to\hat W$.
\end{proof}

\subsection{Examples of Segal spaces}\label{egss}
One of the main application of Segal spaces is to construct an $(\infty,1)$-category structure from a model category. We begin with a slight more general discussion:
\subsubsection{Classifying diagram of categories}
Let $C$ be a category and $W\subset C$ a subcategory such that $\Ob W=\Ob C$. The morphisms in $f\in W$ are called \textbf{weak equivalences}, they are the ones that we want to invert.

\begin{defin}
Let $(C,W)$ be such a pair. Let $I$ be any other category. For any two functors $f,g\in C^I$, we say that a natural transformation $f\xrightarrow{\alpha}g$ is a \textbf{weak equivalence} if $\alpha i \colon f(i) \to g(i)$ is a weak equivalence for all $i\in\Ob I$. Let $\mathrm{we}(C^I)\subset C^I$ be the subcategory of all weak equivalences.

The \textbf{classifying diagram} of $(C,W)$ is defined to be the simplicial space $N(C,W)$ where
\[
N(C,W)_m=\mathrm{nerve}\,\mathrm{we}(C^{[m]}).
\]
where $\mathrm{nerve}(C)$ is the simplicial nerve of a category $C$.
\end{defin}
It is convenient to view an $n$-simplex of $N(C,W)_m$ as a flattened out diagram
\begin{equation}\label{visualncw}
\begin{matrix} c_{00}&\to&\cdots&\to&c_{0m}\\\downarrow&&&&\downarrow\\\vdots&&&&\vdots\\\downarrow&&&&\downarrow\\c_{n0}&\to&\cdots&\to&c_{nm}
\end{matrix}
\end{equation}
where the vertical arrows are weak equivalences.

It follows from the composition of morphisms in an ordinary category that $N(C,W)$ satisfies the Segal condition. However, in general, it is not Reedy-fibrant.

In some special cases they are:
\begin{eg}\label{nerveeg}
\begin{enumerate}
\item Let $C_0\subset C$ be the subcategory consisting of all objects and only the identity morphisms. The \textbf{discrete nerve} $\mathrm{discnerve}\,C=N(C,C_0)$ is a Segal space since all discrete simplicial spaces are Reedy fibrant. In particular $\mathrm{discnerve}([n])=F(n)$. However, it is not complete. Equivalent categories may not give weakly-equivalent discrete nerves, eg., a single point category with only the identity morphism and a two point category with a unique isomorphism between the two points have non-equivalent discrete nerves.

\item Let $\mathrm{iso}\,C\subset C$ be the subcategory consisting of all objects and all invertible morphisms in $C$ (i.e. the maximal subgroupoid of $C$). The \textbf{classifying diagram} $NC=N(C,\mathrm{iso}\,C)$ of $C$ is a complete Segal space. Reedy fibrancy can easily be checked and $NC_{\mathrm{hoequiv}}=\mathrm{nerve}(\mathrm{iso}\,C)=NC_0$. Indeed, $NC=\widehat{\mathrm{discnerve}\,C}$ is a Segal completion of the discrete nerve.

Rezk showed that the functor $C\mapsto NC$ gives rise to a full embedding of categories into complete Segal spaces:
\begin{prop}[{\cite[Thm 3.7, Prop 3.11]{rezk-a-model-for-the-homotopy-theory-of-homotopy-theories}}]\label{nerveprop}
Let $C$ and $D$ be categories. There are natural isomorphisms $N(C\times D)\cong NC\times ND$ and $N(D^C)\cong ND^{NC}$. Hence, the functor $N:\Cat\to \ssp$ is a fully faithful. 

More generally, given $\mathrm{iso}\,D\subset W\subset D$ a subcategory,
$$N(D^C,\mathrm{we}(D^C))\cong N(D,W)^{NC}\cong N(D,W)^{\mathrm{discnerve}\,C}.$$
\end{prop}
\end{enumerate}
\end{eg}

\subsubsection{Classifying diagram of a model category}
Let $C=M$ be a simplicial model category and $W\subset M$ be its subcategory of weak equivalences. As noted above, we have a simplicial space $N(M,W)$. $N(M,W)$ is in general not Reedy fibrant, but we can take a Reedy fibrant replacement $N^f(M,W)$\footnote{As most of the model categories we are interested in are not small, there are some set theoretical considerations. See \cite{rezk-a-model-for-the-homotopy-theory-of-homotopy-theories} for details.}. Define this to be the \textbf{classifying diagram} of the model category $M$.

Using the axioms of model categories, Rezk showed that the Reedy fibrant replacement is not only a Segal space, but is already complete:
\begin{thm}[{\cite[Thm 8.3]{rezk-a-model-for-the-homotopy-theory-of-homotopy-theories}}]
Let $M$ be a simplicial model category. Then, $N^f(M)$ is a complete Segal space.
\end{thm}

While the classifying diagram functor is not fully faithful (different model categories may give rise to the same complete Segal space), in some cases, it does preserve small homotopy limits and colimits. This is given as the strictification theorem for model categories:
\begin{thm}[Strictification theorem, {\cite[Thm 8.11]{rezk-a-model-for-the-homotopy-theory-of-homotopy-theories}}]
Let $I$ and $J$ be small categories. Suppose $M=\sset^J$, then the natural map 
$$f:N(M^I,\mathrm{we}(M^I))\to N^f(M)^{\mathrm{discnerve}\,I}\cong N^f(M)^{NI}$$ 
is a weak equivalence. In particlar, it induces a weak equivalence of complete Segal spaces $N^f(M^I)\to N^f(M)^{NI}$.
\end{thm}
It is expected that this result holds more generally for cofibrantly generated simplicial model categories.

\subsection{Model category structures related to Segal spaces}

To show that complete Segal spaces give a good model for $(\infty,1)$-categories, we need to introduce two other model category structures on the category of simplicial spaces $\ssp$. The class of Reedy weak equivalences is too small and we extend it via left Bousfield localisation (\ref{thm existence of bousfield localization}) to obtain a model category where fibrant-cofibrant objects are complete Segal spaces and the weak equivalences are an appropriate notion of categorical equivalence.

Applying left Bousfield localisation with respect to the set of maps $S=\{G(k)\hookrightarrow F(k)\}$ defining the Segal condition gives the Segal space model structure $\mathcal{SS}$ (see \cite{rezk-a-model-for-the-homotopy-theory-of-homotopy-theories} for details of the proof).
\begin{thm}[{\cite[Thm 7.1]{rezk-a-model-for-the-homotopy-theory-of-homotopy-theories}}]
There exists a model structure, the \textbf{Segal space model structure} $\mathcal{SS}$, where
\begin{enumerate}
\item the cofibrations are the monomorphisms.;
\item the weak equivalences are maps $f$ such that $\Map_{\ssp}(f,W)$ is a weak equivalence for all Segal spaces $W$; and 
\item the fibrations are the maps that satisfy the right lifting property with respect to all trivial cofibrations.
\end{enumerate}
It is proper and compatible with the cartesian closed standard model structure on $\sset$.
\end{thm}

To obtain a complete Segal space model structure $\mathcal{CSS}$, we need to localize $\mathcal{SS}$ with respect to the map $E\to F(0)$ proposed in Cor.~\ref{rephoequivcor}.
\begin{thm}[{\cite[Thm 7.2]{rezk-a-model-for-the-homotopy-theory-of-homotopy-theories}}]
There exists a model structure, the \textbf{complete Segal space model structure} $\mathcal{CSS}$, where
\begin{enumerate}
\item the cofibrations are the monomorphisms.
\item the weak equivalences are maps $f$ such that $\Map_{\ssp}(f,W)$ is a weak equivalence for all complete Segal spaces $W$; and
\item the fibrations are the maps that satisfy the right lifting property with respect to all trivial cofibrations.
\end{enumerate}
It is proper and compatible with the cartesian closed standard model structure on $\sset$.
\end{thm}
It is clear that the fibrant-cofibrant objects are precisely the complete Segal spaces. A Reedy weak equivalence between two objects $U,V$ is a weak equivalence in $\mathcal{CSS}$ while the converse is true if $U,V$ are complete Segal spaces.

We say that a morphism of Segal spaces $g:U\to V$ is a \textbf{categorical equivalence} if there exist morphisms $f,h:V\to U$ such that $g\circ f\sim 1_U$ and $f\circ h\sim 1_V$. The precise definition of the categorical homotopy $\sim$ is given in \cite[Sec.~13.1]{rezk-a-model-for-the-homotopy-theory-of-homotopy-theories}. In a similar way to classical category theory, there is a test for categorical equivalence through fully-faithfulness. This is the notion of Dwyer-Kan equivalence:
\begin{defin}
A map $f:U\to V$ between two Segal spaces is a \textbf{Dwyer-Kan equivalence} if
\begin{enumerate}
\item the induced map $\Ho f:\Ho U\to\Ho V$ is an equivalence of categories; and
\item for each pair of objects $x,x'\in U$, the induced function $\map_U(x,x')\to\map_V(fx,fx')$ is a weak equivalence.
\end{enumerate}
An equivalent formulation of condition (1) is
\begin{description}
\item{1'.} the induced map $\Ob U/\sim\to\Ob V/\sim$ is a bijection on the equivalence classes of objects.
\end{description}
\end{defin}

Rezk \cite[Sec.~7,13,14]{rezk-a-model-for-the-homotopy-theory-of-homotopy-theories} proved that all these notions of equivalences are equivalent for complete Segal spaces.
\begin{thm}\label{thm dk reedy}
Let $f:U\to V$ be a map of Segal spaces. Then $f$ is a Dwyer-Kan equivalence iff $f$ is a weak equivalence in $\mathcal{CSS}$. If, in addition, $U$ and $V$ are complete Segal spaces, then the notions of Dwyer-Kan equivalence, categorical equivalence, Reedy weak equivalence and weak equivalence in $\mathcal{SS}$ and $\mathcal{CSS}$ are all the same.
\end{thm}

This can be summarized in the following diagram for simplicial spaces.
$$\xymatrix{\text{Reedy w.e.}\ar@{=>}@/_.3pc/[r]&\text{w.e. in }\mathcal{SS}\ar@{==>}@/_.3pc/[l]_{\text{SS}}\ar@{=>}@/_.3pc/[r]&\text{w.e. in }\mathcal{CSS}\ar@{==>}@/_.3pc/[l]_{\text{CSS}}\ar@{<=>}[r]_{\text{SS}}&\text{DK-equiv.}\ar@{==>}@/_.3pc/[r]_{\text{CSS}}&\text{cat. equiv.}\ar@{=>}@/_.3pc/[l]}$$

Rezk also proved that the functorial completion map $i_W:W\to\widehat W$ is a Dwyer-Kan equivalence and a weak-equivalence in $\mathcal{CSS}$. This means that very often it suffices to work with Segal spaces and Dwyer-Kan equivalences, since the completion map is difficult to compute and most naturally occurring Segal spaces are not complete (eg. bordism categories \cite{lurie-classification-topological-field-theories}).

\section{Segal Categories}
While complete Segal spaces are good models for weak $(\infty,1)$-categories, the difficulty in computing the completion map necessitates an alternative model in some cases. Segal categories avoid the cumbersome procedure of completion, allowing for explicit constructions (without using the small object argument), but the compromise is that we require the 0-space to be discrete. It is the preferred construction for many applications such as higher stacks.

Segal categories were formally defined by Dwyer, Kan and Smith in \cite{dwyer-kan-smith-homotopy-commutative-diagrams-and-their-realizations}. They were used extensively and generalised to $n$-Segal categories by Hirschowitz and Simpson in their studies of $n$-stacks \cite{hirschowitz-simpson-descente-pour-les-n-champs}. In this section, we will follow the ideas in \cite{hirschowitz-simpson-descente-pour-les-n-champs} but refer to the work of Bergner \cite{bergner-three-models-for-the-homotopy-theory-of-homotopy-theories} for more explicit constructions.

\subsection{Segal precategories, categories and closed model structure}
\begin{defin}
A simplicial space $X$ is a Segal precategory if $X_0$ is discrete. Let $\mathbf{PCat}$ be the category of Segal pre-categories. A Segal precategory $X$ that satisfies the Segal condition (\ref{eq seg con}) is called a Segal category.
\end{defin}

\begin{eg}
Let $C$ be any category, then $\mathrm{discnerve}\,C$ is a Segal category. Note that under the model category structure that we are going to impose on $\mathbf{PCat}$, $f:C\to D$ is an equivalence of categories iff $\mathrm{discnerve}\,f$ is a weak equivalence (unlike in the Reedy model structure or $\mathcal{SS}$).
\end{eg}

Every simplicial enriched category $C$ can be seen as a Segal precategory, which we will also denote by $C$, by setting $C_0=\Ob C$ and $C_n=\sqcup_{x,y\in\Ob C}(\map_C(x,y))_{n-1}$ with appropriate face and degeneracy maps. Note that, however, a Segal category may not be a simplicial category since associativity is not strict. To obtain a simplicial category, we will need to consider the category generated by the Segal category, which we will not define here.

Recall that given a simplicial space $X$, we can define $\Ob X$ and mapping spaces $\map_X(x,y)$ (Def.~\ref{simpcat}). We denote $x\sim y$ if $\map_X(x,y)\ne\emptyset$. However, $\sim$ may not be an equivalence relation. We denote by the same symbol the equivalence relation generated by $\sim$.

The model structure on Segal precategories corresponding to Segal categories cannot be given by any Bousfield localisation. Instead, it requires some effort to define a proper notion of weak equivalences. We first construct a functor $L_C:\mathbf{PCat}\to\mathbf{PCat}$ that sends a precategory $X$ into a Segal category $L_CX$. Hirschowitz and Simpsons proved the existence of such a functor (indeed for Segal $n$-categories) in \cite{hirschowitz-simpson-descente-pour-les-n-champs} but we shall present the explicit construction given by Bergner \cite{bergner-three-models-for-the-homotopy-theory-of-homotopy-theories}.

We want to construct $L_CX$ as a functorial fibrant replacement of $X$ in the Segal space model category structure $\mathcal{SS}$, in such a way that $L_CX$ is still a precategory. We proceed by the small object argument. There is a set of generating trivial cofibrations in $\mathcal{SS}$ (obtained from the generating trivial cofibrations of the Reedy model structure by localisation):
$$F(k)\times\Lambda^l_t\sqcup_{G(k)\times\Lambda^l_t}G(k)\times\Delta^l\to F(k)\times\Delta^l,\qquad k\ge0,l\ge1,0\le t\le l$$
where $G(0)=\emptyset$. Define $L_CX$ to be the colimit of the iterated pushouts
$$\xymatrix{\coprod(F(k)\times\Lambda^l_t\sqcup_{G(k)\times\Lambda^l_t}G(k)\times\Delta^l)\ar[r]\ar[d]&X_i\ar[d]\\
\coprod(F(k)\times\Delta^l)\ar[r]&X_{i+1}}$$
where $X=X_0$ and the coproduct is taken over all $k> 0$, $l\ge 1$ and $0\le t\le l$. Note that the generating trivial cofibrations with $k=0$ are excluded since they affect the discreteness of the 0-space of $X$.

\begin{prop}
$L_CX$ as defined above is a functorial fibrant replacement of $X$, so $L_C:\mathbf{PCat}\to\mathbf{PCat}$ is a well-defined functor taking a precategory to a Segal category that is also a Segal space.
\end{prop}
\begin{proof}
Note that the map on the 0-space induced by a generating trivial cofibration is given by $[k]\times\Lambda^l_t\cup[k]\times\Delta^l\cong [k]\times\Delta^l\xrightarrow{\id}[k]\times\Delta^l$ for $k> 0$, so $L_CX$ has discrete 0-space and is a Segal category.

The small object argument states that the colimit of the iterated pushouts with respect to all generating trivial cofibrations give a fibrant replacement functor, so it suffices to check that $L_CX$ satisfies the RLP with respect to all generating trivial cofibrations with $k=0$. This is equivalent to checking that the lift exists in the following diagram
$$\xymatrix{\Lambda^l_t\ar[r]\ar[d]&\Map_{\ssp}(F(0),L_CX)\cong (L_CX)_0\\\Delta^l\ar@{-->}[ur]}.$$
This is true since $L_CX$ is a discrete simplicial set and hence a Kan complex.
\end{proof}

%%We remark that in the case where $f:X\to Y$ is a map between two Segal categories which are also Segal spaces, then the two definitions of Dwyer-Kan equivalence are equivalent.

We are now ready to define a model category structure on precategories.

\begin{thm}\label{thm pcat model}
There exists a closed model category structure on $\mathbf{PCat}$ in which
\begin{enumerate}
\item the cofibrations are precisely the monomorphisms;
\item the weak equivalences are precisely the maps $f:X\to Y$ such that $L_Cf:L_CX\to L_CY$ is a Dwyer-Kan equivalence of Segal spaces; and
\item the fibrations are maps that satisfy the right lifting property with respect to all trivial cofibrations.
\end{enumerate}
We denote $\mathbf{PCat}$ equipped with this model category structure $\mathcal{SC}$. The fibrant-cofibrant objects of this model structure are precisely the Reedy-fibrant Segal categories.
\end{thm}
\begin{proof}
See \cite[Thm 2.3]{hirschowitz-simpson-descente-pour-les-n-champs} and \cite[Thm 5.1]{bergner-three-models-for-the-homotopy-theory-of-homotopy-theories} for two different proofs of the existence of this model structure. See \cite[Cor 5.13]{bergner-three-models-for-the-homotopy-theory-of-homotopy-theories} and \cite[Thm 3.2]{bergner-characterization-of-fibrant-segal-categories} for the proof of the last statement.
\end{proof}

Thm.~\ref{thm pcat model} and the comparisons between the different models of $(\infty,1)$-categories (Sec.~\ref{sec comp thms}) show that Reedy-fibrant Segal categories give a good model of $(\infty,1)$-categories. Reedy-fibrant Segal categories occur more naturally than complete Segal spaces, and is often preferred in geometrical constructions.

Let $\mathrm{SeCat}\subset\mathcal{SC}$ be the full subcategory of Segal categories, which are not required to be Reedy-fibrant. Since $\mathrm{SeCat}$ contains all fibrant-cofibrant objects, $\Ho\mathrm{SeCat}\cong\Ho\mathcal{SC}$ and in particular contains all small homotopy limits and colimits.

\subsection{Segal localisation}
Let $C$ be a category and $W\subset C$ be a subcategory of weak equivalences, we want a construction similar to the classifying diagram construction in complete Segal spaces. This is done through a localisation procedure, described in \cite{toen-tamsamani-categories-segal-categories-and-applications}. The diagram of categories
\begin{equation} \label{seglocalpushout}
\xymatrix{\coprod_{f\in W}[1]\ar[r]^-{\sqcup i}\ar[d]&\coprod_{f\in W}I[1]\\C}
\end{equation}
induces a pushout square in $\mathcal{SC}$ of the discrete nerves
$$\xymatrix{\coprod_{f\in W}F(1)\ar[r]\ar[d]&\coprod_{f\in W}E\ar[d]\\\mathrm{discnerve}\,C\ar[r]&L(C,W)}.$$
Since the top arrow is a cofibration and all objects are cofibrant, it is a homotopy pushout square as well. Hence, we may choose $L(C,W)\in\mathrm{SeCat}$ (indeed, we may even choose $L(C,W)\in\mathcal{SC}_{cf}$ the subcategory of Reedy-fibrant Segal categories).

If $W=\mathrm{iso}\,C$, we get $LC=\mathrm{discnerve}\,C$.

The \textbf{Segal localisation} $L(C,W)$ is related to the classifying diagram $N(C,W)$ by the following proposition.

\begin{prop}
There is a natural morphism $N(C,W)\to L(C,W)$ that is a Reedy weak equivalence. Hence, their Segal completions are $\widehat{N(C,W)}$ and $\widehat{L(C,W)}$ are naturally equivalent as complete Segal spaces.
\end{prop}
\begin{proof}
Let $N([1],[1])$ be a classifying space of $C=[1]$ with all morphisms regarded as weak equivalences. Then, it is clear that the following is a homotopy pushout square:
$$\xymatrix{\coprod_{f\in W}F(1)\ar[r]^{\sqcup N(i)}\ar[d]&\coprod_{f\in W}N([1],[1])\ar[d]\\NC\ar[r]&N(C,W)}.$$
The natural inclusion $N([1],[1])\subset N(I[1])$ is a Reedy weak equivalence. This is because, for every $m\ge 0$, $\mathrm{nerve}\,\mathrm{we}([1]^{[m]})\to\mathrm{nerve}\,\mathrm{iso}(I[1]^{[m]})$ induces a weak equivalence since their geometric realisations are both contractible.

Hence, the homotopy pushout induces a Reedy weak equivalence $N(C,W)\xrightarrow\sim\hat L(C,W)$.
\end{proof}

In particular, for a simplicial model category $M$, we have $N^f(M)=\widehat{L(M,W)}$.

In \cite{dwyer-kan-calculating-simplicial-localizations}, Dwyer and Kan showed that $L(C,W)$ can be explicitly constructed using the hammock localisation $L^H(C,W)$ (see also Expos\'e~\ref{chapter:simplicial-localization}). So for a simplicial model category $M$, Dwyer and Kan proved that

\begin{thm}[{\cite{dwyer-kan-function-complexes-in-homotopical-algebra}}]
Let $M$ be a simplicial model category. Then, $LM=L(M,W)\cong L^HM\cong M_{cf}$ are weakly equivalent.
\end{thm}

This theorem allows us to compute the localisation of simplicial model categories easily.
\begin{eg}
\begin{enumerate}
\item Let $M=\sset$ be the category of simplicial sets with the usual model structure, then $M_{cf}$ is the subcategory of Kan complexes. $M_{cf}$ satisfies the Segal condition since homotopy is an equivalence relation on Kan complexes. Hence, $L\sset$ gives us the theory of Kan complexes (or equivalently CW-complexes) as an $(\infty,1)$-category.
\item Let $M=\mathrm{Top}$ be the category of topological spaces, then $L\mathrm{Top}\cong\mathrm{Top}_{cf}$ is the subcategory of CW-complexes. Hence, we see that the localisation of $\sset$ and $\mathrm{Top}$ give the same Segal category.
\item Let $M=\ssp$ with the Reedy model structure. We thus get $L\ssp$ as the Segal category of all Reedy-fibrant simplicial spaces. If we equip the simplicial spaces with the (complete) Segal space model structure, we get $L\mathcal{SS}$ ($L\mathcal{CSS}$) the Segal category of all (complete) Segal spaces.
\item Let $M=\mathcal{SC}$, then $L\mathcal{SC}$ is the Segal category of all Segal categories (up to some change in universe).
\end{enumerate}
\end{eg}
We see that Segal localisation gives us a $(\infty,1)$-category of $(\infty,1)$-categories. As in the case of complete Segal spaces, we have a strictification theorem.

\begin{thm}[Strictification theorem]
Let $M$ be a cofibrantly generated simplicial model category and $I$ a small category, then we have a weak equivalence of Segal categories (Dwyer-Kan equivalence)
$$L(M^I)\cong L(M)^{\mathrm{discnerve}\,I}.$$
\end{thm}
\begin{proof}
See \cite{hirschowitz-simpson-descente-pour-les-n-champs} and \cite{toen-vezzosi-segal-topoi-and-stacks-over-segal-categories}.
\end{proof}

So, to compute small limits and colimits of (complete) Segal spaces or Segal categories, it suffices to compute the homotopy limits and colimits in the larger categories of simplicial spaces or precategories. The strictification theorem also gives a form of Yoneda's lemma for $(\infty,1)$-categories. See To\"en and Vezzosi's paper \cite{toen-vezzosi-segal-topoi-and-stacks-over-segal-categories} for more details.


\section{Comparison theorems}\label{sec comp thms}
In this section, we will show that complete Segal spaces, Reedy-fibrant Segal categories and the previously introduced quasi-categories are equally valid models for $(\infty,1)$-categories. To this end, we will state a number of comparison theorems between complete Segal spaces, Segal categories and other models of $(\infty,1)$-categories.

The main mechanism in proving the equivalence of two model categories is Quillen equivalence (Sec.~\ref{subsec Quillen adjunction}). Recall that a Quillen equivalence of model categories induces an adjoint equivalence of the underlying homotopy categories (Thm.~\ref{thm quillen adjuntion}).
This implies that if we have a Quillen equivalence between two models of infinity categories, they have the same objects up to weak equivalences and their full subcategories of fibrant-cofibrant objects are homotopy equivalent (which is the best we can expect to ask for).

We have the following diagram of Quillen equivalences.
$$\xymatrix{\mathcal{CSS}\ar@/^/[dr]&\mathcal{SC}\ar[l]\ar@/^/[d]&\mathcal{SC}'\ar[l]\ar[r]&s\mathcal{C}\\
&\mathcal{QC}\ar@/^/[ul]\ar@/^/[u]\ar[urr]}$$
An arrow in the diagram refers to the direction of the left adjoint. $\mathcal{QC}$ is the category of simplicial sets with the Joyal model category structure, $s\mathcal{C}$ is the category of simplicial enriched categories with an associated model structure and $\mathcal{SC}'$ is another model structure on precategories (see \cite{bergner-characterization-of-fibrant-segal-categories,joyal-quasicategories-vs-simplicial-categories} for the definitions).

We shall only present the construction of some of the Quillen equivalences, omitting all the proofs. Good references include \cite{barwick-schommer-pries-unicity-homotopy-theory,bergner-characterization-of-fibrant-segal-categories,joyal-quasicategories-vs-simplicial-categories,joyal-tierney-quasi-categories-vs-segal-spaces}.
Consider the projection and inclusion functors (on the first component) $p_1:\Delta\times\Delta\to\Delta:([m],[n])\mapsto[m]$ and $i_1:\Delta\to\Delta\times\Delta:[n]\mapsto([n],0)$. They induce an adjoint pair of functors
$$p_1^*:\sset\rightleftarrows\sset^{(2)}:i_1^*$$
between simplicial sets and bisimplicial sets. Under the identification of bisimplicial sets with simplicial spaces, $p_1^*$ sends a simplicial set $X$ into a discrete simplicial space $\tilde X$ with $\tilde X_m$ being the discrete simplicial set generated by $X_m$. $i_1^*$ associates a simplicial space $Y$ with the simplicial set with $n$-simplices given by $(Y_n)_0$. These functors induce a Quillen equivalence
\begin{thm}\cite[Sec.~4]{joyal-tierney-quasi-categories-vs-segal-spaces}
The adjoint pair
$$p_1^*:\mathcal{QC}\rightleftarrows\mathcal{CSS}:i_1^*$$
is a Quillen equivalence.
\end{thm}

Let $\Delta^{|2}=([0]\times\Delta)^{-1}(\Delta\times\Delta)$ where we formally invert all morphisms in $[0]\times\Delta$. There is a canonical map $\pi:\Delta\times\Delta\to\Delta^{|2}$. Since $p_1:\Delta\times\Delta\to\Delta$ sends all morphisms in $[0]\times\Delta$ to invertible morphisms, it factors through $q:\Delta^{|2}\to\Delta$ where $q\pi=p_1$. Define $j=\pi i_1:\Delta\to\Delta^{|2}$. Then $q$ and $j$ restrict $p_1$ and $i_1$ to $\Delta^{|2}$ and induce an adjoint pair of functors
$$q^*:\sset\rightleftarrows\mathbf{PCat}:j^*.$$
Explicitly, $q^*$ takes a simplicial set $X$ to the discrete bisimplicial space $\tilde X$ with $\tilde X_m$ being the discrete simplicial set generated by $X_m$. $j^*$ takes a precategory $Y$ to the simplicial set $Y_{*0}$.

\begin{thm}\cite[Sec.~5]{joyal-tierney-quasi-categories-vs-segal-spaces}
The adjoint pair
$$q^*:\mathcal{QC}\rightleftarrows\mathcal{SC}:j^*$$
is a Quillen equivalence.
\end{thm}

We now demonstrate a Quillen equivalence between $\mathcal{CSS}$ and $\mathcal{SC}$. There is a natural inclusion of precategories into simplicial spaces $I:\mathcal{SC}\to\mathcal{CSS}$. We can construct a right adjoint, the discretization functor $R:\mathcal{CSS}\to\mathcal{SC}$ defined by the homotopy pullback square
$$\xymatrix{RW\ar[r]\ar[d]&\mathrm{cosk}(W_{0,0})\ar[d]\\W\ar[r]&\mathrm{cosk(W_0)}},$$
that is we take the discretization of the 0-space of $W$. If $W$ is a complete Segal space, we can explicitly define $RW$ by $RW_0=W_{0,0}$ is a discrete simplicial set, $RW_1$ by the homotopy pullback
$$\xymatrix{RW_1\ar[r]\ar[d]&W_{0,0}\times W_{0,0}\ar[d]\\W_1\ar[r]&W_0\times W_0}$$
and $RW_k=RW_1\times_{RW_0}\cdots\times_{RW_0}RW_1$ for $k\ge2$.
\begin{thm}\cite{bergner-characterization-of-fibrant-segal-categories}
The adjoint pair
$$I:\mathcal{SC}\rightleftarrows\mathcal{CSS}:R$$
is a Quillen equivalence.
\end{thm}

Note that $I=\pi^*$ as defined above, so we have a commutative triangle $p_1^*=Iq^*$.

With these three Quillen equivalences, we can thus conclude that the categories of quasi-categories, complete Segal spaces and Reedy-fibrant Segal categories are homotopy equivalent.

\printbibliography[heading = local]

\end{refsection}
